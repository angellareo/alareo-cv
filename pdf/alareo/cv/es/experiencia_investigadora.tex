%-------------------------------------------------------------------------------
%	SECTION TITLE
%-------------------------------------------------------------------------------
\cvsection{Méritos de investigación y transferencia}

\cvsubsection{Participación en proyectos de investigación}
%-------------------------------------------------------------------------------
%	CONTENT
%-------------------------------------------------------------------------------
\begin{cventries}
%%---------------------------------------------------------
\cventry
{Doctor en Ingeniería Informática y Telecomunicaciones} % Job title
{Sinergias entre Neurocomputación e Inteligencia Artificial: Una perspectiva integral} % Organization
{Escuela Politécnica Superior. Universidad Autónoma de Madrid.} % Location
{Sep. 2024 - Actualidad} % Date(s)
{
	\begin{cvitems} % Description(s) of tasks/responsibilities
		\item {MICIU PID2023-149669NB-I00 - IPs: Prof. Francisco de Borja Rodríguez Ortiz, Luis Fernando Lago Fernández. }
	\end{cvitems}
}

%%---------------------------------------------------------
  \cventry
    {Doctor en Ingeniería Informática y Telecomunicaciones} % Job title
    {BryoMicroClim - Combining microclimate sensor networks and models to uncover the vulnerability of small plants to climate change} % Organization
    {BIOPOLIS-CIBIO, University of Porto.} % Location
    {Ene. 2023 - Actualidad} % Date(s)
    {
      \begin{cvitems} % Description(s) of tasks/responsibilities
      	\item {\href{https://bryomicroclim.net/en/}{\underline{Web del proyecto: https://bryomicroclim.net}} }
        \item {2022.03116.PTDC. - IP: Prof. Helena Hespanhol.}
      \end{cvitems}
    }

%%---------------------------------------------------------
  \cventry
    {Investigador predoctoral} % Job title
    {Incorporación de principios de procesamiento de información naturales en el diseño de algoritmos de computación artificial.} % Organization
    {Escuela Politécnica Superior. Universidad Autónoma de Madrid.} % Location
    {Ene. 2021 - Dic. 2023} % Date(s)
    {
      \begin{cvitems} % Description(s) of tasks/responsibilities
        \item {MICINN PID2020-114867RB-I00.}
        \item {IP: Prof. Francisco B. Rodriguez.}
      \end{cvitems}
    }

%%---------------------------------------------------------
  \cventry
    {Investigador predoctoral} % Job title
    {SCENIC - Escalando los efectos de las dinámicas de nicho e interacciones en las consecuencias ecológicas y evolutivas de la coexistencia.} % Organization
    {Facultad de Ciencias. Universidad Autónoma de Madrid.} % Location
    {Ene. 2020 - May. 2023} % Date(s)
    {
      \begin{cvitems} % Description(s) of tasks/responsibilities
        \item {MICINN PID2019-106840GA-C22.}
        \item {IP: Prof. Nagore García Medina}
        \item {Diseño e implementación de un \underline{\href{https://github.com/united-ecology/btmboard}{\textit{datalogger} de temperatura y humedad}} (Arduino).}
      \end{cvitems}
    }

%%---------------------------------------------------------
  \cventry %https://fundacion-biodiversidad.es/sites/default/files/multimedia/archivos/13_resolucion_politicas_publicas_2019.pdf
    {Coautor} % Job title
    {Un año para aplicar el Acuerdo de Paris.} % Organization
    {Ecologistas en Acción} % Location
    {Sep. 2019 - Dic. 2019} % Date(s)
    {
      \begin{cvitems} % Description(s) of tasks/responsibilities
      	\item {Ayuda concedida por la Fundación Biodiversidad (MITRD).}
        \item {Beneficiario: Ecologistas en Acción.}
        \item {Diseño del modelo y desarrollo del software (Python) para los análisis del informe. \underline{\href{https://www.ecologistasenaccion.org/132893/informe-escenarios-de-trabajo-en-la-transicion-ecosocial-2020-2030/}{Escenarios de trabajo en la transición ecosocial (2020-2030)}} }
      \end{cvitems}
    }

%%---------------------------------------------------------
  \cventry
    {Investigador predoctoral} % Job title
    {Interacción dinámica entre sistemas computación natural y sistemas artificiales} % Organization
    {Escuela Politécnica Superior. Universidad Autónoma de Madrid.} % Location
    {Ene 2018 - Dic.  2020} % Date(s)
    {
      \begin{cvitems} % Description(s) of tasks/responsibilities
        \item {MINECO TIN2017-84452-R.}
        \item {IP: Prof. Francisco B. Rodriguez.}
        \item {Desarrollo de \underline{\href{https://github.com/GNB-UAM/electromotor-nmodel}{modelo computacional de la red electromotora}} (C++).}
      \end{cvitems}
    }
    
%  \cventry
%    {Predoctoral researcher} % Job title
%    {New Bioinspired Paradigms Applied to Artificial Computing.} % Organization
%    {Madrid, Spain} % Location
%    {Jun. 2017 - May. 2018} % Date(s)
%    {
%      \begin{cvitems} % Description(s) of tasks/responsibilities
%        \item {MINECO TIN2010-19607. IP: Francisco B. Rodriguez.}
%      \end{cvitems}
%    }

%%---------------------------------------------------------

  \cventry
    {Investigador predoctoral} % Job title
    {Estancia de investigación. Beca Iberoamérica Santander Investigación.} % Organization
    {Puebla, Mexico.} % Location
    {Aug. 2016 - May. 2017} % Date(s)
    {
    \begin{cvitems}
     \item {Proyecto: \textbf{Modelo neuronal de descarga eléctrica en peces eléctricos de la especie \textit{Gnathonemus petersii}}.}
     \item {Departamento de Ciencia Computational. Instituto Tecnológico y de Estudios Superiores de Monterrey (ITESM).}
     \item {Tutor: Prof. Alberto Oliart}
    \end{cvitems}
    }

%%---------------------------------------------------------

  \cventry
    {Investigador predoctoral} % Job title
    {Estudio y análisis del procesamiento dinámico de la información en sistemas de computación naturales y bioinspirados.} % Organization
    {Escuela Politécnica Superior. Universidad Autónoma de Madrid.} % Location
    {Ene. 2015 - Dic. 2017} % Date(s)
    {
      \begin{cvitems} % Description(s) of tasks/responsibilities
        \item {MINECO TIN2014-54580-R.}
        \item {IP: Prof. Francisco B. Rodriguez.}
        \item {Implementación de \underline{\href{http://arantxa.ii.uam.es/~gnb/material.htm}{software en tiempo real para estimulación en ciclo cerrado de sistema vivo}} (C/RTAI para implementar módulo del kernel de Linux).}
      \end{cvitems}
    }

\end{cventries}

%-------------------------------------------------------------------------------
%	PUBLICACIONES
%-------------------------------------------------------------------------------
% IF JCR				https://factor.recursoscientificos.fecyt.es/
% SJR 					https://www.scimagojr.com/
% SCOPUS - CiteScore	https://www.scopus.com/sources
%-------------------------------------------------------------------------------

\cvsubsection{Publicaciones en revistas científicas}

\begin{itemize}
\item \fullcite{ayala2024neural}
	
\item \fullcite{lareo2022modeling} \textbf{(IF-JCR-2022: 3.739, 15/55  Mat. and Computational Biology (Q2) // IF-SJR-2022: 0.975, 52/292, Biomedical Engineering (Q1))}

\item \fullcite{lallana2021assessing} \textbf{(IF-JCR-2021: 3.889, 133/279 Environmental Sciences (Q2) // IF-SJR-2021: 0.66, 166/756, Geography, Planning and Development (Q1))}

\item \fullcite{leo2019btm} \textbf{(IF-JCR-2019: 1,163, 45/71 Multidisciplinary Sciences (Q3) // IF-SJR-2019: 0.57, 97/166 Neuroscience (Q2))}

\item \fullcite{lareo2016temporal} \textbf{(IF-JCR-2016: 3.87, 6/57 Mat. and Computational Biology (Q1) // IF-SJR-2016: 2.437, 7/416, Biomedical Engineering (Q1))}
\end{itemize}


%-------------------------------------------------------------------------------
%	CONFERENCIAS
%-------------------------------------------------------------------------------
\cvsubsection{Divulgación de los resultados de la actividad investigadora}

\textbf{Publicaciones de congreso}
\begin{itemize}
 \item \fullcite{ayala2024parameterization}
 \item \fullcite{hespanhol2024microclimate} 
 \item \fullcite{ayala2023matching} \textbf{(IF-SJR-2023: 0.242, 95/141 Information Systems and Management (Q3))} % AIAI2023
 \item \fullcite{lareo2023assessing} \textbf{(IF-JCR-2021: 1.5, 46/65 Mat. and Computational Biology (Q3))} %CNS2022
 \item \fullcite{ayala2023closedloop} \textbf{(IF-JCR-2021: 1.5, 46/65 Mat. and Computational Biology (Q3))} %CNS2022
 \item \fullcite{lareo2021closedloop} \textbf{(IF-JCR-2021: 1.453, 50/57 Mat. and Computational Biology (Q4))}
 \item \fullcite{ayala2021closedloop} \textbf{(IF-JCR-2021: 1.453, 50/57 Mat. and Computational Biology (Q4))}
 \item \fullcite{lareo2019tuning} \textbf{(IF-JCR-2019: 1.5, 46/65 Mat. and Computational Biology (Q3))}
 \item \fullcite{lareo2018evolutionary} \textbf{(SCOPUS CiteScore-2018: 1.6, 95/211 General Computer Science (Q2))}
 \item \fullcite{lareo2017analysis} \textbf{(SCOPUS CiteScore-2017: 1.6, 84/208 General Computer Science (Q2))}
 \item \fullcite{lareo2017informationtheoretic} (IF-JCR-2017: 2.173, 185/261, Q3)
 \item \fullcite{forlim2017closedloop} (IF-JCR-2017: 2.173, 185/261, Q3)
 \item \fullcite{cobos2016open} \textbf{(IF-SJR-2017: 1.453, 50/57 Mat. and Computational Biology (Q4))}
\end{itemize}


\textbf{Participaciones en congresos}
% To search conference impact:
% https://portal.core.edu.au/conf-ranks/ - ICORE
% 
\begin{itemize}
 \item Oral presentation: \fullcite{hespanhol2024bryomicroclimconfa}
 \item Oral presentation: \fullcite{hespanhol2024bryomicroclimconf}
 \item Oral presentation: \fullcite{hespanhol2024microclimateconf}
 \item Oral presentation: \fullcite{ayala2024parameterizationconf}
 \item Poster presentation: \fullcite{ayala2023identifyingconf} 
 \item Poster presentation: \fullcite{lareo2022assessingconf} 
 \item Poster presentation: \fullcite{ayala2022closedloopconf}
 \item Poster presentation: \fullcite{ayala2021closedloopconf}
 \item Poster presentation: \fullcite{lareo2021closedloopconf}
 \item Oral presentation: \fullcite{moreno2021importance} 
 \item Poster presentation: \fullcite{lareo2019tuningconf}
 \item Oral presentation: \fullcite{lareo2018evolutionaryconf}
 \item Oral presentation: \fullcite{lareo2017analysisconf} 
 \item Poster presentation\fullcite{lareo2017informationtheoreticconf}
 \item Poster presentation: \fullcite{forlim2017closedloopconf}
 \item Oral \& poster presentation: \fullcite{lareo2016sequential}
 \item Oral presentation: \fullcite{lareo2014weakly}
\end{itemize}

\textbf{Organización de congresos}
\begin{cventries}
  \cventry
    {27th International Conference on Artificial Neural Networks}
    {Program Committee Member ICANN 2018}
    {Rhodes (Greece)}
    {5‐7 Oct. 2018}
	{\begin{cvitems}
			\item Reviewer: \fullcite{horzyk2018associative}
	 \end{cvitems}	
	}
\end{cventries}

\textbf{Otros eventos de divulgación científica}
\begin{cventries}
  \cventry
    {Participante:  \href{https://microbiosdigital.com/20/08/03/bioscience-storytelling-challenge/}{Bioscience Storytelling Challenge}} % Affiliation/role
    {Concurso de storytelling científico} % Organization/group
    {Online} % Location
    {Jul. 2020} % Date(s)
    {
    \begin{cvitems} % Description(s) of experience/contributions/knowledge
        \item {\underline{\href{https://microbiosdigital.com/2020/08/03/bioscience-storytelling-challenge/}{Página web del evento}}.}        	
    \end{cvitems}
    }
    
  \cventry
    {Participante:  \href{https://workshops.ift.uam-csic.es/cambioclimatico/Programa}{El cambio climático contado por expertos}} % Affiliation/role
    {Presentación: ¿Afectará a tu empleo el acuerdo de París?} % Organization/group
    {Universidad Autónoma de Madrid} % Location
    {Dec. 2019} % Date(s)
    {
    \begin{cvitems} % Description(s) of experience/contributions/knowledge
        \item {\underline{\href{https://m.youtube.com/watch?v=rB3-6bywW_Q}{Video de la presentación}}.}        	\end{cvitems}
    }
\end{cventries}